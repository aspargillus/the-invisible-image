\documentclass[12pt]{article}
\usepackage[english,german]{babel}
\usepackage[latin1]{inputenc}
\usepackage{rotating}
%\usepackage{pstricks}
\usepackage{textpos}
\usepackage{color}
\usepackage{vmargin}
\pagestyle{empty}
\setpapersize{A4}
\setmargnohfrb{15mm}{20mm}{25mm}{20mm} % l t r b
\renewcommand{\familydefault}{\sfdefault}
\newlength{\Indent}
\setlength{\Indent}{30mm}
\parindent=0.8em
\parskip=0.25em
\rightskip=0pt plus 3em
\leftskip=\Indent
\setcounter{secnumdepth}{-2}

\makeatletter
\newcommand{\English}{\selectlanguage{english}\slshape}
\newcommand{\Deutsch}{\selectlanguage{german}\upshape}
\renewcommand\section{\@startsection {section}{1}{-\Indent}%
                                   {-3.5ex \@plus -1ex \@minus -.2ex}%
                                   {2.3ex \@plus.2ex}%
                                   {\sffamily\normalfont\Large}}
\makeatother
%\newcommand{\Section}[2]{%
%  \section{\protect\parbox{\linewidth}{}}
\newcommand{\Section}[2]{\section{%
    \protect\parbox{\linewidth}{%
    {\bfseries\Deutsch#1}
   \\% \hskip1em%
    {\English#2}}}}

\begin{document}
% \begin{sideways}
% \begin{minipage}{\linewidth}
%   \ttfamily%
%   \LARGE%
%   {
%     %\Huge
%     \fontsize{36}{42}\selectfont%
%     the invisible image\\[0.66\baselineskip]
%   }
%   by \\
%   Frank~Schimmel \& Christopher~Clay
% \end{minipage}
% \end{sideways}
% \newpage
%\vskip-2\baselineskip%
\Section{Was sind Spacer GIFs?}{What are Spacer GIFs?}
%\makebox[]{\rput{0}(0,0){test}}
%\setlength{\TPHorizModule}{\textwidth}%
%\setlength{\TPVertModule}{\textheight}%
\definecolor{mygray}{rgb}{0.66,0.66,0.66}%
\begin{textblock*}{0pt}(-\Indent,0\textheight)%
  \begin{sideways}%
    \Large\ttfamily%
    \begin{minipage}{.9\textheight}
    \textcolor{mygray}{%
      {\Huge%
        the invisible image}\\{}
      [\kern-.125em[second interation, for network glimpse]\kern-.125em]\\{}
    by Frank~Schimmel \& Christohper Clay}
    \end{minipage}
  \end{sideways}
\end{textblock*}%

\Deutsch%
\emph{Spacer GIFs} -- auch \emph{Blind GIFs} genannt, obwohl es sich
dabei auch um andere Grafikformate handeln kann -- sind ein
Anachonismus auf den fr�hen Tagen des \emph{World Wide Web}: In den
fr�hen bis Mitte der 1990er Jahre, also vor der Einf�hrung von
\emph{Cascading Style Sheets} (CSS), waren die Mittel zur optischen
Gestaltung von \emph{Websites} weitaus eingeschr�nkter als heute.
Insbesondere die genaue Positionierung der einzelnen Elemente einer
Seite war nur bedingt m�glich. Ein solches Mittel waren die
\emph{Spacer GIFs}. Sie haben bei der Positionierung von
Seitenelementen geholfen, indem sie Teile der Seite freigehalten
haben, um so z.B.  zwischen Eintr�gen in Navigationsmen�s oder den
Spalten eines mehrspaltigen Layouts einen angemessenen Abstand zu
bewahren.

Vom technischen Standpunkt aus betrachtet sind diese augenscheinlichen
L�cken jedoch alles andere als leer, werden sie doch von Grafiken
ausgef�llt, die lediglich unsichtbar sind. So lebt eine ganze Spezies
von Bildern unbemerkt direkt vor den Augen nichtsahnender Besucher der
betreffenden \emph{Websites}.

Eine weitere bemerkenswerte Eigenschaft von \emph{Spacer GIFs} ist,
da� sie einen \emph{Hack} im klassischen Sinn des Wortes darstellen:
Sie sind ein Ge(Mis?)brauch (Schaffung von Leere auf einer Seite)
eines Mittels (Einbettung von Grafiken in Textdokumenten), das f�r
einen g�nzlich anderen Zweck (Illustrationen und Abbildungen)
erschaffen wurde.

Und sie sind eine akut vom Aussterben bedrohte Spezies: Die heute zur
Gestaltung von \emph{Websites} zur Verf�gung stehenden Mittel sind
ihnen bei weitem �berlegen und sie k�nnen nur noch als �berholt
angesehen werden. Diese Installation ist diesen Bildern gewidmet, die
sonst unbemerkt in ihren Verstecken hausen und bietet die seltene
Gelegenheit, die Letzten ihrer Art zu bestaunen.


\par\vskip0.8\baselineskip\noindent
\English%
Spacer GIFs---also called blind GIFs though they may as well be of
other graphics formats---are an anachronism from the early days of the
world wide web: back in the early to mid 1990s, before the advent of
cascading style sheets (CSS), means for specifying the layout of a web
site---especially ensuring the exact positioning of page
elements---were quite limited. And one of them were spacer GIFs.
Their purpose was to help position other elements of the page by
keeping portions of that page empty, e.g. separating entries of a
navigation menu or text columns in a multi-column layout.

Technically, though, these apparently empty spaces are not empty at
all. They are occupied by images which just happen to be invisible, a
whole species unnoticed by unsuspecting visitors of the sites they
dwell on.

Another remarkable propertiy of spacer GIFs is that they constitute a
Hack in the classical sense of the word: they are a very clever
(ab?)use (generating empty space in a page) of a means (embedding
images in a text document) invented for an entirely different purpose
(providing illustrations and figures).

They also are a species endangered by extinction: todays means of
specifying page layout are vastly superior, thus, rendering spacer
GIFs obsolete. This installation is dedicated to these normally unseen
images, dragging them out of their niches and hideaways before they
die out completely. So this may be your Last Chance (not) To See.


\Section{Funktionsweise.}{How it works.}

\Deutsch%
Kern der Installation ist ein Programm, das die Google-Bildersuche
benutzt, um das WWW nach \emph{Spacer}n abzusuchen. Genauer gesagt,
wird nach Grafiken mit Namen gesucht, die mit relativ hoher
Wahrscheinlichkeit \emph{Spacer GIFs} sind. Die gefundenen Grafiken
werden heruntergeladen und auf Transparenz gepr�ft. Wenn die
vollst�ndig unsichtbar sind, werden sie dann dargestellt.

Mit den Kn�pfen k�nnen uneingeschr�nkte oder spezialisierte Suchen
gestartet, zum n�chsten Bild der aktuellen Suche gesprungen bzw. die
schon fr�her gefundenen \emph{Spacer} revidiert werden.

\par\vskip0.8\baselineskip\noindent%
\English%
At the core of the installation is a program that searches the web for
spacers using Google Images. More precisely, it searches for images
with names that likely will return many spacer GIFs. It then downloads
the images and tests them for transparency. If it finds them to be
entirely invisible, it will display them.

By pressing the different buttons provided, you can execute an
unrestricted or specialised search, force the next image of the
current search or review spacers found so far.


\Section{Was (nicht) zu sehen ist.}{What you see here. (Or don't.)}

\Deutsch%
W�hrend das Programm nach weiteren Bildern sucht, zeigt es seinen
Fortschritt dabei auf dem Monitor. Sobald ein \emph{Spacer GIF}
gefunden wurde, werden dessen Metadaten wie die seine URL, Ausma�e in
Pixeln, Gr��e in Bytes und dergleichen in der Projektion gezeigt.
Danach wird das Bild selbst projeziert. Gleichzeitig wird die Struktur
einer Seite, die das Bild referenziert schematisch auf dem Monitor
dargestellt. Dabei werden die Seitenelemente als graue Bl�cke
gezeigt. Je tiefer diese in der Hierarchie verschachtelt sind, desto
heller erscheinen sie. Das \emph{Spacer GIF} selbst erschein rot.

\par\vskip0.8\baselineskip\noindent%
\English%
While it searches for more invisible images, the program shows its
progress on the monitor. Once a spacer GIF has been found you are
shown the images meta-data, i.e. its URL, dimensions in pixels, size in
Bytes etc. in the projection. It then switches to display the image
itself. In parallel, the layout structure of a page, which references
the image, in is shown on the monitor. Page elements are only shown as
gray blocks. The deeper nested the element, the brighter it is shown.
The spacer GIF itself is appearing in red.




\newpage
{
\Large%
\setlength{\fboxsep}{3mm}%
\newcommand{\Label}[2]{%
\fbox{%
  \Deutsch%
  Uneingeschr�nkte Suche
  \\
  \English%
  Unrestricted search
}
}

\fbox{\parbox{6cm}{%
  \Deutsch%
  Uneingeschr�nkte Suche
  \\
  \English%
  Unrestricted Search
}}

\vskip2\baselineskip

\fbox{\parbox{5cm}{%
  \Deutsch%
  Spezialisierte Suchen
  \\
  \English%
  Specialised Search
}}

\vskip2\baselineskip

\fbox{\parbox{5cm}{%
  \Deutsch%
  N�chstes Bild
  \\
  \English%
  Next Image
}}

\vskip2\baselineskip

\fbox{\parbox{8cm}{%
  \Deutsch%
  Bekannte Bilder aus dem Cache
  \\
  \English%
  Known Images From The Cache
}
}}


\end{document}
